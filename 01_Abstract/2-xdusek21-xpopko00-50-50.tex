\documentclass[a4paper,12pt]{article}
\usepackage[left=2.5cm,right=2cm,top=3cm]{geometry}
\usepackage[utf8x]{inputenc}
\usepackage[czech]{babel}
\usepackage[IL2]{fontenc}
\usepackage{eso-pic}
\usepackage{graphicx}

\renewcommand{\baselinestretch}{1.2}

\begin{document}

	% Logo FIT
	\AddToShipoutPictureBG{
		\AtPageUpperLeft{\raisebox{-\height}{\includegraphics[scale=0.50]{fit-logo.pdf}}}
	}
	
	% Turn off numbering
	\pagenumbering{gobble}	

	\setlength{\parindent}{0pt}
	\vspace*{10pt}
	\LARGE \textsc{Abstrakt}
	\normalsize

	\vspace*{5pt}
	\textit{Projekt do předmětu SIN - Inteligentní systémy} \\
	\textit{Téma č. 2, Subsystém inteligentní budovy} \\
	\textit{Řešitelé: Daniel Dušek (xdusek21), Anna Popková (xpopko00)}

	\setlength{\parindent}{15pt}
	\setlength{\parskip}{15pt}
	\renewcommand{\baselinestretch}{1.5}
	\vspace*{15pt}
	V projektu bude navržen model podsystému inteligentní budovy a realizován systém řízení tohoto podsystému. Návrh a řízení se bude týkat místnosti -- ložnice, ve které budou umístěny světelné, tepelné, vlhkostní a pohybové senzory. Cílem bude vytvářet optimální prostředí pro spánek, popř. denní činnosti, ke kterým může být ložnice využita. 

	Dle geografické pozice a času budou zatahovány světlo nepropouštějící rolety, které budou zajišťovat, že každý den bude ve stejný čas v místnosti odpovídající úroveň osvětlení dle uživatelsky navolených časů. Obráceně pak ráno budou žaluzie buď automaticky roztahovány, aby propustili světlo v čase vstávání do pokoje, nebo zůstanou zatažené (v případě brzkého rozednění) a postupně bude přidávána intenzita umělého chladného osvětlení.

	Subsytém bude také sledovat vlhost v místnosti a dle její úrovně spouštět zvhlčovač vzduchu. V období zimy bude taktéž automaticky regulovat teplotu v místnosti. Pohybové senzory v místnosti budou sloužit k detekci probuzení uživatele, aby v případě nutnosti opustit ložnici v průběhu noci mohly být rozsvíceny jemné, červené diody umožňující lepší orientaci ve tmě bez nutnosti přivykání očí na světlo.
	
\end{document}